\documentclass{article}

% Comments follow the % sign

% Packages you may need in your TeX document to include figures, pseudocode etc.
\usepackage{kantlipsum}
\usepackage{graphicx}
\usepackage{algorithm}
\usepackage{algorithmic}
\usepackage{mathtools}
\usepackage{xcolor}
\usepackage{listings}
\definecolor{graycode}{rgb}{0.5,0.5,0.5}
\definecolor{background}{RGB}{230,230,230}

\lstset{basicstyle=\ttfamily,
  backgroundcolor=\color{background},
  numberstyle=\small\color{graycode},
  showstringspaces=false,
  keywordstyle=\color{blue},
  morekeywords={"dateRep", "day", "month", "year", "cases", "deaths", "countriesAndTerritories", "geoId", "countryterritoryCode", "popData2020", "continentExp", CREATE, TABLE, INDEX, ON, INSERT, INTO, SELECT, DISTINCT, FROM, WHERE, AS, INNER, JOIN, ASC, ORDER, BY, GROUP, LIMIT},
  otherkeywords={"dateRep", "day", "month", "year", "cases", "deaths", "countriesAndTerritories", "geoId", "countryterritoryCode", "popData2020", "continentExp", CREATE, TABLE, INDEX, ON, INSERT, INTO, SELECT, DISTINCT, FROM, WHERE, AS, INNER, JOIN, ASC, ORDER, BY, GROUP, LIMIT},
  breaklines=true,
  commentstyle=\color{red},
  numbers=left,
  numbersep=6pt
}

\title{COMP1204: Data Management \\ Coursework Two}
\author{Lukas Kakogiannos \\ lk1u20, 32158998}

\begin{document}
\maketitle

\begin{abstract}
    The aim of this coursework is to create an SQLite based database to represent data of the ongoing Coronavirus outbreak from an Open Data Source \textbf{dataset.csv}. Moreover, the database model should be fully normalised, facilitating querying.
\end{abstract}

\section{The Relational Model}

\subsection{EX1}

The table below shows the relations directly represented in the dataset.csv file, and their respective data types.

\begin{table}[ht]
    \caption{Relations Within the Dataset}
    \vspace{5pt}
    \centering
    \begin{tabular}{c c}
    \hline\hline
      CovidData  & SQLite Attribute \\ [0.5ex]
    \hline
      dateRep  & TEXT \\
      day  & INTEGER \\
      month  & INTEGER \\
      year  & INTEGER \\
      cases  & INTEGER \\
      deaths  & INTEGER \\
      countriesAndTerritories  & TEXT \\
      geoId  & TEXT \\
      countryterritoryCode  & TEXT \\
      popData2020  & INTEGER \\
      continentExp  & TEXT \\ [0.5ex]
    \hline
    \end{tabular}
    \label{tab:relations}
\end{table}

\subsection{EX2}

After examining the dataset, the following functional dependencies can be observed:

\begin{enumerate}
    \item $dateRep \rightarrow  day$
    \item $dateRep \rightarrow  month$
    \item $dateRep \rightarrow  year$
    \item $day, month, year \rightarrow  dateRep$
    \item $countriesAndTerritories \rightarrow  geoId$
    \item $geoId \rightarrow  countryterritoryCode$
    \item $geoId \rightarrow  popData2020$
    \item $geoId \rightarrow  continentExp$
    \item $geoId \rightarrow  countriesAndTerritories$
    \item $countryterritoryCode \rightarrow  geoId$
    \item $dateRep, countriesAndTerritories \rightarrow  cases$
    \item $dateRep, countriesAndTerritories \rightarrow  deaths$
\end{enumerate}

\subsection{EX3}

All potential candidate keys:

\begin{enumerate}
    \item dateRep, countriesAndTerritories
    \item dateRep, geoId
    \item day, month, year, countriesAndTerritories
    \item day, month, year, geoId
\end{enumerate}

\subsection{EX4}
After comparing all candidate keys, I chose \textbf{\{dateRep, geoId\}} to be an optimal primary key as it has the least attributes.

\section{Normalisation}
\subsection{EX5}
    
The partial-key dependencies present in the relation are listed below:

\begin{center}
    \{day, month, year\} partially dependant to \{dateRep\} \\
    \{geoId, countryterritoryCode, popData2020, continentExp\} partially dependant to \{countriesAndTerritories\}
\end{center}
    
Hence, as part of the decomposition, the following relations could be deduced:

\begin{center}
Date\{dateRep, day, month, year\} \\
CountryInformation\{countriesAndTerritories, geoId, countryterritoryCode, popData2020, continentExp\}
\end{center}

\subsection{EX6}
By using the answer to EX5, we can convert the relation to 2nd Normal Form.

\begin{table}[ht]
    \caption{Date}
    \vspace{5pt}
    \centering
    \begin{tabular}{c c c}
    \hline\hline
      CovidData  & SQLite Attribute & Key\\ [0.5ex]
    \hline
      dateRep  & TEXT & -primaryKey\\
      day  & INTEGER \\
      month  & INTEGER \\
      year  & INTEGER \\ [0.5ex]
    \hline
    \end{tabular}
    \label{tab:date}
\end{table}

\begin{table}[ht]
    \caption{CountryInformation}
    \vspace{5pt}
    \centering
    \begin{tabular}{c c c}
    \hline\hline
      CovidData  & SQLite Attribute & Key\\ [0.5ex]
    \hline
      countriesAndTerritories  & TEXT & -primaryKey\\
      geoId  & TEXT \\
      countryterritoryCode  & TEXT \\
      popData2020  & INTEGER \\
      continentExp  & TEXT \\ [0.5ex]
    \hline
    \end{tabular}
    \label{tab:countryInfo}
\end{table}

By introducing these new relations, it is possible to reduce the main relation to the following:

\begin{table}[ht]
    \caption{ReducedCovidData}
    \vspace{5pt}
    \centering
    \begin{tabular}{c c c}
    \hline\hline
      CovidData  & SQLite Attribute & Key\\ [0.5ex]
    \hline
      dateRep  & TEXT & -foreignKey\\
      cases  & INTEGER \\
      deaths  & INTEGER \\
      countriesAndTerritories  & TEXT & -foreignKey\\ [0.5ex]
    \hline
    \end{tabular}
    \label{tab:Reduced}
\end{table}

\newpage

\subsection{EX7}

By looking at the dataset, it is possible to assume that the population of a country is a constant number, therefore, the new relations contain one transitive dependency:

\begin{center}
    $countriesAndTerritories \rightarrow countryterritoryCode \rightarrow popData2020$
\end{center}

\subsection{EX8}

Based on the dependency found in EX7, a new relation has to be formed in order to achieve 3rd Normal Form. Assuming every country has a geoId, the following relation can be formed:

\begin{table}[ht]
    \caption{Population}
    \vspace{5pt}
    \centering
    \begin{tabular}{c c c}
    \hline\hline
      CovidData  & SQLite Attribute & Key\\ [0.5ex]
    \hline
      geoId  & INTEGER & -primaryKey\\
      popData2020  & INTEGER \\ [0.5ex]
    \hline
    \end{tabular}
    \label{tab:Population}
\end{table}

Finally, \textbf{countryInformation} will change to:

\begin{table}[ht]
    \caption{ReducedCountryInformation}
    \vspace{5pt}
    \centering
    \begin{tabular}{c c c}
    \hline\hline
      CovidData  & SQLite Attribute & Key\\ [0.5ex]
    \hline
      countriesAndTerritories  & TEXT & -primaryKey\\
      geoId  & TEXT & -foreignKey\\
      countryterritoryCode  & TEXT \\
      continentExp  & TEXT \\ [0.5ex]
    \hline
    \end{tabular}
    \label{tab:countryInfo2}
\end{table}

\subsection{EX9}

For each one of the four relations (ReducedCovidData, Date, ReducedCountryInformation, Population), each one of their dependencies X → Y is either a superkey or a trivial functional dependency ($Y \subseteq X$), consequently, all of them are already in Boyce-Codd Normal Form.

\section{Modelling}
\subsection{EX10}

To start modelling the database physically, I first had to import the raw data from \textbf{dataset.csv} into a table called \textbf{dataset} in a SQLite database called \textbf{coronavirus.db}. To achieve that, the following commands were used:

\begin{lstlisting}[caption={CMD commands for dataset.sql}]
sqlite3 coronavirus.db

CREATE TABLE dataset( "dateRep" TEXT, "day" INTEGER, "month" INTEGER, "year" INTEGER, "cases" INTEGER, "deaths" INTEGER, "countriesAndTerritories" TEXT, "geoId" TEXT, "countryterritoryCode" TEXT, "popData2020" INTEGER, "continentExp" TEXT );

sqlite> .mode csv
sqlite> .import dataset.csv dataset
sqlite> .output dataset.sql
sqlite> .dump dataset
\end{lstlisting}

\subsection{EX11}

After creating \textbf{dataset.sql}, I wrote the SQL to represent the four relations mentioned in EX9(ReducedCovidData, Date, ReducedCountryInformation, Population) as additional tables. This was achieved with the use of \textbf{CREATE} statements and \textbf{CONSTRAINT} statements for the primary/foreign keys. Furthermore, the six following indexes were added to help querying.

\begin{lstlisting}[caption={Indexes for querying}]
CREATE INDEX idx_ReducedCovidData_cases
ON ReducedCovidData(cases);

CREATE INDEX idx_ReducedCovidData_deaths
ON ReducedCovidData(deaths);

CREATE INDEX idx_ReducedCovidData_dateRep_cases
ON ReducedCovidData(dateRep, cases);

CREATE INDEX idx_ReducedCovidData_dateRep_deaths
ON ReducedCovidData(dateRep, deaths);

CREATE INDEX idx_ReducedCovidData_countriesAndTerritories_cases
ON ReducedCovidData(countriesAndTerritories, cases);

CREATE INDEX idx_ReducedCovidData_countriesAndTerritories_deaths
ON ReducedCovidData(countriesAndTerritories, deaths);
\end{lstlisting}

\subsection{EX12}

By using \textbf{INSERT INTO} and \textbf{SELECT DISTINCT} statements, the new tables were populated with values from the 'dataset' table. To exclude the first line of the database, the following statement was used:

\begin{center}
    WHERE dateRep != 'dateRep'
\end{center}

Find below a full example of a final set of statements to populate the \textbf{Date} table:

\begin{lstlisting}[caption={Date Table Statements}]
INSERT INTO Date(dateRep, day, month, year)
SELECT DISTINCT dateRep, day, month, year
FROM dataset WHERE dateRep != 'dateRep';
\end{lstlisting}

\subsection{EX13}

After several tests and tables gone wrong, the desired result was achieved, a fully populated database and its respective tables. 
Moreover, the import and dump functions were used similarly to EX10 in order to dump the database unto \textbf{dataset2.sql} and \textbf{dataset3.sql}, after running \textbf{EX11.sql} and \textbf{EX12.sql}, respectively.

\section{Querying}
\subsection{EX14}

To get the worldwide total number of cases and deaths, the following statement was used:

\begin{lstlisting}[caption={Querying worldwide total number of cases and deaths}]
SELECT SUM(cases) AS 'total cases', SUM(deaths) AS 'total deaths'
FROM ReducedCovidData;
\end{lstlisting}

\subsection{EX15}

To get the number of cases by date, in increasing date order, for the United Kingdom, the following statement was used:

\begin{lstlisting}[caption={Querying number of cases for the UK by increasing date}]
SELECT ReducedCovidData.dateRep AS 'Date', cases AS 'Number_Of_Cases'
FROM ReducedCovidData INNER JOIN Date ON ReducedCovidData.dateRep = Date.dateRep
WHERE countriesAndTerritories = 'United_Kingdom'
ORDER BY year ASC, month ASC, day ASC;
\end{lstlisting}

\subsection{EX16}

To get the number of cases and deaths by date, in increasing date order, for each country, the following statement was used:

\begin{lstlisting}[caption={Querying number of cases and deaths for all countries by increasing date}]
SELECT ReducedCovidData.countriesAndTerritories AS 'Country', ReducedCovidData.dateRep AS 'Date', SUM(cases) AS 'Number_Of_Cases', SUM(deaths) AS 'Number_Of_Deaths'
FROM ReducedCovidData
	INNER JOIN Date ON ReducedCovidData.dateRep = Date.dateRep
	INNER JOIN ReducedCountryInformation ON ReducedCovidData.countriesAndTerritories = ReducedCountryInformation.countriesAndTerritories
GROUP BY ReducedCovidData.dateRep, ReducedCovidData.countriesAndTerritories
ORDER BY year ASC, month ASC, day ASC;
\end{lstlisting}

It can be seen from the \textbf{GROUP BY} statement that the result is ordered by date first.

\subsection{EX17}

To get the total number of cases and deaths as a percentage of the population rounded to three decimal cases, for each country, the following statement was used:

\begin{lstlisting}[caption={Querying total number of cases and deaths of a country(percentage of population)}]
SELECT ReducedCovidData.countriesAndTerritories AS 'Country', ROUND((SUM(cases) * 100.0) / (Population.popData2020), 3) AS '%_Cases_of_Population', ROUND((SUM(deaths) * 100.0) / (Population.popData2020), 3)AS '%_Deaths_of_Population'
FROM ReducedCovidData
	INNER JOIN ReducedCountryInformation ON ReducedCovidData.countriesAndTerritories = ReducedCountryInformation.countriesAndTerritories
	INNER JOIN Population ON ReducedCountryInformation.geoId = Population.geoId
GROUP BY ReducedCovidData.countriesAndTerritories;
\end{lstlisting}

\subsection{EX18}

To get a descending list of the the top 10 countries, by percentage total deaths out of total cases in that country rounded to three decimal cases, the following statement was used:

\begin{lstlisting}[caption={Querying 10 countries with highest deaths to cases ratio percentage}]
SELECT ReducedCovidData.countriesAndTerritories AS 'Country', ROUND((SUM(deaths) * 100.0) / (SUM(cases) * 1.0), 3) AS '%_Deaths_of_Country_Cases'
FROM ReducedCovidData
GROUP BY countriesAndTerritories
ORDER BY 2 DESC
LIMIT 10;
\end{lstlisting}

\subsection{EX19}

To get the date against a cumulative running total of the number of deaths by day and cases by day for the United Kingdom, the following statement was used:

\begin{lstlisting}[caption={Querying cumulative number of deaths and cases over time for the UK}]
SELECT ReducedCovidData.dateRep AS 'Date', SUM(cases) OVER(ORDER BY year ASC, month ASC, day ASC) AS 'Cumulative_UK_cases', SUM(deaths) OVER(ORDER BY year ASC, month ASC, day ASC) AS 'Cumulative_UK_deaths'
FROM ReducedCovidData
	INNER JOIN Date ON ReducedCovidData.dateRep = Date.dateRep
WHERE countriesAndTerritories = 'United_Kingdom'
ORDER BY year ASC, month ASC, day ASC;
\end{lstlisting}

\end{document}
